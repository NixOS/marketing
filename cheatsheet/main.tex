\documentclass[landscape, 10pt]{article}
\input{paper.tex}
\input{color.tex}
\usepackage{multicol}
\usepackage[most]{tcolorbox}
\usepackage{svg}

\usepackage[hidelinks]{hyperref}
\hypersetup{
    colorlinks=true,
    urlcolor=nixdarkblue,
}

% Set typewriter font everywhere
\usepackage[T1]{fontenc}
\renewcommand{\familydefault}{\ttdefault}

% Declare Unicode characters
\DeclareUnicodeCharacter{21D2}{$\Rightarrow$}

% Get rid of red boxes around minted "errors"
\AtBeginEnvironment{minted}{%
    \renewcommand{\fcolorbox}[4][]{#4}%
}

% Create the box that is used everywhere
\newtcolorbox{ctexpression}[2][]{
    enhanced, 
    breakable,
    colback=codelight,
    colframe=nixdarkblue,
    before skip=1mm, after skip=1mm,left=1mm, right=1mm, top=1mm, bottom=1mm,
    attach boxed title to top left={yshift*=-\tcboxedtitleheight}, 
    boxed title size=title,
    boxed title style={%
        sharp corners, 
        rounded corners=northwest, 
        colback=tcbcolframe, 
        boxrule=0pt,
    },
    underlay boxed title={%
        \path[fill=tcbcolframe]
            (title.south west)--
            (title.south east)
            to [out=0, in=180] ([xshift=5mm]title.east)--
            (title.center-|frame.east)
            [rounded corners] |- (frame.north) -|
            cycle;
    },fonttitle=\bfseries, title={#2},#1
}

% Shrink the top and bottom separation around minted environments
\newlength{\fancyvrbtopsep}
\newlength{\fancyvrbpartopsep}
\makeatletter
\FV@AddToHook{\FV@ListParameterHook}{\topsep=\fancyvrbtopsep\partopsep=\fancyvrbpartopsep}
\makeatother
\setlength{\fancyvrbtopsep}{0pt}
\setlength{\fancyvrbpartopsep}{0pt}

\newenvironment{tightcenter}{%
  \setlength\topsep{0pt}
  \setlength\parskip{0pt}
  \begin{center}
}{%
  \end{center}
}

\begin{document}

\renewcommand{\arraystretch}{1.5}

% Set the background image
\AddToHook{shipout/background}{%
    \put (0in,-0.9\paperheight){\scalebox{-1}[1]{\includesvg[width=\paperwidth,height=\paperheight]{background.svg}}}%
}

\begin{multicols*}{3}

\begin{ctexpression}{Types \& Syntax}
    \begin{tabular}{l p{0.5\textwidth}}
    String &
        \begin{minipage}{0.5\textwidth}
            \begin{minted}{nix}
"this is a string"
''
  a multi-line string
''
            \end{minted}
        \end{minipage}
        \\
        % \hdashline
        Boolean &
        \begin{minipage}{0.5\textwidth}
            \begin{minted}{nix}
true, false
            \end{minted}
        \end{minipage}
        \\
        % \hdashline
        Null &
        \begin{minipage}{0.5\textwidth}
            \begin{minted}{nix}
null
            \end{minted}
        \end{minipage}
        \\
        % \hdashline
        Integer &
        \begin{minipage}{0.5\textwidth}
            \begin{minted}{nix}
123, -123
            \end{minted}
        \end{minipage}
        \\
        % \hdashline
        Float &
        \begin{minipage}{0.5\textwidth}
            \begin{minted}{nix}
3.14, -3.14
            \end{minted}
        \end{minipage}
        \\
        % \hdashline
        Path &
        \begin{minipage}{0.5\textwidth}
            \begin{minted}{nix}
/an/absolute/path
./a/relative/path
            \end{minted}
        \end{minipage}
        \\
        % \hdashline
        Attribute Set &
        \begin{minipage}{0.5\textwidth}
            \begin{minted}{nix}
{ a = 1; b = 2; }
            \end{minted}
        \end{minipage}
        \\
        % \hdashline
        List &
        \begin{minipage}{0.5\textwidth}
            \begin{minted}{nix}
[ 1 2.0 "thing" null ]
            \end{minted}
        \end{minipage}
        \\
        % \hdashline
        Comment &
        \begin{minipage}{0.5\textwidth}
            \begin{minted}{nix}
# A single line comment.
/*
  A multi-line comment.
*/
            \end{minted}
        \end{minipage}
    \end{tabular}
\end{ctexpression}

\vfill

\begin{ctexpression}{Scoping}
{\color{nixdarkblue}Define local variables using} let...in
    \begin{minted}{nix}
let
  x = 1;
  y = 2;
in
  x + y # ⇒ returns 3
    \end{minted}
    \tcbline
{\color{nixdarkblue}Add variables into scope using} inherit
    \begin{minted}{nix}
let x = 1; in { inherit x; }
    \end{minted}
{\color{nixdarkblue}desugars to}
    \begin{minted}{nix}
let x = 1; in { x = x; }
    \end{minted}
    \tcbline
{\color{nixdarkblue}Add attributes into scope using} inherit
    \begin{minted}{nix}
let x = { y = 1; }; in { inherit (x) y; }
    \end{minted}
{\color{nixdarkblue}desugars to}
    \begin{minted}{nix}
let x = { y = 1; }; in { y = x.y; }
    \end{minted}
    \tcbline
{\color{nixdarkblue}Add all attributes into scope using} with
    \begin{minted}{nix}
let
  s = { x = 1; y = 2; };
in
  with s;
  x + y # ⇒ returns 3
    \end{minted}
\end{ctexpression}

\begin{ctexpression}{Conditionals}
{\color{nixdarkblue}Define conditionals using} if...then...else
    \begin{minted}{nix}
if x > 0
then 1
else -1
    \end{minted}
\end{ctexpression}

\vfill

\begin{ctexpression}{Attribute Sets}
{\color{nixdarkblue}Update attribute sets using} //
    \begin{minted}{nix}
{ x = 1; } // { y = 2; }
# ⇒ returns { x = 1; y = 2;}
{ x = 1; } // { x = 2; }
# ⇒ returns { x = 2; }
    \end{minted}
    \tcbline
{\color{nixdarkblue}Check if an attribute exists using} ?
    \begin{minted}{nix}
let
  s = { x = 1; };
in
  s ? x # ⇒ returns true
    \end{minted}
    \tcbline
{\color{nixdarkblue}Reference attribute keys using} .
    \begin{minted}{nix}
let
  s = { x = { y = 1; }; };
in
  s.x.y # ⇒ returns 1
    \end{minted}
    \tcbline
{\color{nixdarkblue}Reference optional attribute keys using} or
    \begin{minted}{nix}
let
  s = { x = 1; };
in
  s.y or 2 # ⇒ returns 2
    \end{minted}
\end{ctexpression}

\vfill

\begin{ctexpression}{Concatenation \& Interpolation}
{\color{nixdarkblue}Concatenate lists using} ++
    \begin{minted}{nix}
[ 1 2 ] ++ [ 3 4 ] # ⇒ returns [ 1 2 3 4 ]
    \end{minted}
    \tcbline
{\color{nixdarkblue}Concatenate paths and strings use} +
    \begin{minted}{nix}
/bin + /sh # ⇒ returns /bin/sh
/bin + "/sh" # ⇒ returns /bin/sh
"/bin" + "/sh" # ⇒ returns "/bin/sh"
    \end{minted}
    \tcbline
{\color{nixdarkblue}Interpolate strings using} \$\string{\string}
    \begin{minted}{nix}
let
  x = "bar";
in
  "foo ${x} baz" # ⇒ returns "foo bar baz"
    \end{minted}
\end{ctexpression}

\begin{ctexpression}{Functions}
{\color{nixdarkblue}Simple function}
    \begin{minted}{nix}
let
  f = x: x + 1;
in
  f 1 # ⇒ returns 2
    \end{minted}
    \tcbline
{\color{nixdarkblue}Multiple arguments}
    \begin{minted}{nix}
let
  f = x: y: [ x y ];
in
  f 1 2 # ⇒ returns [ 1 2 ]
    \end{minted}
    \tcbline
{\color{nixdarkblue}Named arguments with attribute sets}
    \begin{minted}{nix}
let
  f = { x, y }: x + y;
in
  f { x=1; y=2; } # ⇒ returns 3
    \end{minted}
    \tcbline
{\color{nixdarkblue}Named arguments; ignores extras using} ...
    \begin{minted}{nix}
let
  f = { x, y, ... }: x + y;
in
  f { x=1; y=2; z=3; } # ⇒ returns 3
    \end{minted}
    \tcbline
{\color{nixdarkblue}Named arguments; default values using} ?
    \begin{minted}{nix}
let
  f = { x, y ? 2 }: x + y;
in
  f { x=1; } # ⇒ returns 3
    \end{minted}
    \tcbline
{\color{nixdarkblue}Bind set arguments to a variable using} @
    \begin{minted}{nix}
let
  f = { x, y }@args: args.x + args.y;
in
  f { x=1; y=2; } # ⇒ returns 3
    \end{minted}
\end{ctexpression}

\vfill

{\setlength\parindent{0pt}
\includesvg[width=\columnwidth]{nixos.svg}

\begin{tightcenter}
{\large Official Nix Language Cheat Sheet}
\end{tightcenter}

\vfill

Additional Resources:

{\small
\href{https://nixos.org/manual/nix/stable/language/builtins}{nixos.org/manual/nix/stable/language/builtins}

\href{https://nixos.org/manual/nixpkgs/stable/\#id-1.4}{nixos.org/manual/nixpkgs/stable/\#id-1.4}
}

\vspace{0.2cm}
}

\end{multicols*}

\end{document}

